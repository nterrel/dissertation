Machine learning has revolutionized computational chemistry by providing efficient and scalable methods for predicting molecular properties. 
Among these, the ANAKIN-ME (ANI) neural network potentials offer a balance of speed and accuracy, enabling large-scale atomistic simulations with near-quantum accuracy. 
This dissertation explores the application of ANI models in molecular simulations, with a focus on predictive reliability, uncertainty quantification, and large-scale implementation in molecular dynamics.

The first chapter introduces machine learning in chemistry and provides background on ANI neural network potentials. 
Chapter 2 examines atomic energy predictions in ANI models, probing their uncertainty at the atomistic level.
Initial efforts to quantify uncertainty in atomic energies proved unsuccessful, leading to a shift toward atomic force uncertainty, which is the focus of Chapter 3. 
This chapter investigates force prediction uncertainty, seeking a physically meaningful uncertainty metric that can be applied to molecular simulations.

Chapter 4 details the development of LAMMPS-ANI, an interface enabling large-scale molecular dynamics simulations using ANI potentials within the Large-scale Atomic/Molecular Massively Parallel Simulator (LAMMPS). 
This framework allows researchers to efficiently model reactive molecular systems at high temperatures while maintaining quantum-mechanical accuracy.

The scalability of LAMMPS-ANI was demonstrated in a hero run, discussed in Chapter 5, where an early Earth simulation of 22.8 million atoms at 2500 K was performed. 
This simulation generated an unprecedented volume of data, necessitating GPU-accelerated analysis pipelines to identify and characterize the millions of novel molecules formed under extreme conditions. 
The challenges of handling big data in large-scale molecular simulations, as well as the emergence of complex chemical networks, are explored.

The dissertation concludes with a discussion of ANI's impact on computational chemistry, the challenges of scaling machine learning potentials to massive simulations, and future directions for improving uncertainty quantification and high-throughput molecular discovery. 
By advancing both the methodological and computational frameworks for ANI-based simulations, this work contributes to the ongoing effort to bridge the gap between quantum accuracy and large-scale molecular modeling.