Machine learning has revolutionized computational chemistry by providing efficient and scalable methods for predicting molecular properties. 
Among these, the ANAKIN-ME (ANI) neural network potentials offer a balance of speed and accuracy, enabling large-scale atomistic simulations with near-quantum accuracy. 
This dissertation explores the application of ANI models in molecular simulations, with a focus on predictive reliability, uncertainty quantification, and large-scale implementation in molecular dynamics.

The first chapter introduces machine learning in chemistry and provides background on ANI neural network potentials. 
Chapter 2 examines energy predictions in ANI models, explores the atomistic properties of ANI predictions, and the uncertainty associated with those predictions.
Initial efforts to quantify uncertainty in atomic energies proved unsuccessful, leading to a shift toward uncertainty in atomic force predictions, which is the focus of Chapter 3. 
This chapter investigates force prediction uncertainty, seeking a physically meaningful uncertainty metric that can be applied to molecular simulations.
Chapter 3 also introduces a new uncertainty-based approach to sampling substructures in a framework called LUKE: Use the Forces.

Chapter 4 details LAMMPS-ANI, an interface enabling large-scale molecular dynamics simulations using ANI potentials within the Large-scale Atomic/Molecular Massively Parallel Simulator (LAMMPS). 
This framework allows for efficient modeling of reactive molecular systems at high temperatures while maintaining near-quantum-mechanical accuracy.
The scalability of LAMMPS-ANI is demonstrated, culminating in a Hero Run, discussed in Chapter 5, where an early Earth simulation of 22.8 million atoms at 2500 K was performed. 
This simulation generated a massive volume of data, necessitating GPU-accelerated analysis pipelines to identify and characterize the millions of novel molecules formed under extreme conditions. 
The challenges of handling big data in large-scale molecular simulations, as well as the emergence of complex chemical diversity, are explored throughout Chapter 5.

This dissertation concludes with a discussion of ANI's impact on computational chemistry, the challenges of scaling machine learning potentials to massive simulations, and future directions for improving uncertainty quantification and high-throughput molecular discovery. 
By advancing both the methodological and computational frameworks for ANI-based simulations, this work contributes to the ongoing effort to reconcile the chemical accuracy of quantum mechanics with large-scale molecular modeling.