\authorRemark{Below is the abstract from my qualifying exam}\\
Using machine learning models to predict chemical properties and potential energy surfaces has become an important complement to traditional approaches to computation and simulation. The Artificial NeurAl networK engINe for Molecular Energies (ANI) methodology has proven useful in producing a transferable neural network potential (NNP) capable of predicting molecular energy within 1 kcal/mol of the reference quantum mechanical (QM) level of theory used in training, at a greatly reduced computational cost. The datasets used in training the ANI neural network potentials were generated with active learning via query by committee (QBC) such that the ensembled neural networks can predict molecular energies with reasonable agreement between ensemble members. Despite an agreement of total energy, the atomic neural networks used in predicting atomic energy contributions to the total energy can yield highly varying outputs. The ANI methodology lacks a means of identifying specific environments in which atomic energy contributions are predicted with high variance by the ensemble. Recognizing patterns in ensemble uncertainty for atomic energy contributions is hypothesized to give insight on the ability of individual atomic networks to understand different atomic environments of varying complexity. This proposal outlines an approach to identifying and correcting deficiencies in the machine-learned potential energy surface produced by the ANI methodology for atomic energy contributions which are predicted with high variance by the ensemble of neural network potentials. 