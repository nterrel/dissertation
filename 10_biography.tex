Nicholas Terrel graduated with a Bachelor of Science degree in Chemistry and a minor in Astronomy from Truman State University.
He gained his first research experiences in his Observational Astronomy course and went on to do computational chemistry research in Dr. Bill Miller's lab.
In the Miller lab, Nicholas learned the basics of Bash scripting and was introduced to the AMBER molecular dynamics suite, though his interests were more materials-focused than the computational biochemistry that the research lab specialized in.
After a short time, and with the guidance of Dr. Miller, he proposed the carbon nanotube project: simulating water transport and filtration using size-exclusive carbon nanotubes.
Working under Dr. Miller, Nicholas developed a passion for computational methods and physical chemistry, and went on to look for graduate programs with researchers focusing on data science and machine learning of chemical systems.

This search led him to the research lab of Dr. Adrian Roitberg, and their work with the ANAKIN-ME neural network potentials. 
Here, Nicholas contributed to the open source TorchANI and LAMMPS-ANI software packages, focusing on the measurement and minimization of empirical uncertainty for the first half of his PhD, followed by applications of machine-learned interatomic potential models to molecular dynamics simulations in the tail-end of his graduate career.
During his time in the Roitberg Lab, Nicholas was afforded several opportunities to present his research in regional and national conferences, including the American Chemical Society Florida Section and National conferences, the Scientific Python (SciPy) conference, and symposia hosted by the University of Florida: AI Days and the 64th Sanibel Symposium. 

Outside of research, coursework, and teaching responsibilities, Nicholas served for three years, 2022-2024, as a Graduate Ambassador for the Computational Theory division of the Chemistry Department, welcoming prospective PhD students to the University of Florida.
He was selected for this position by the graduate coordinator as a representative of the division of Physical Chemistry and Computational Theory, and of the department as a whole.
This volunteer position involved planning the visit weekend for prospective graduate students and carried many responsibilities, including overseeing the coordination of the social meet and greet between current and prospective PhD students.

Nicholas enjoys spending his free time traveling, especially to state and national parks to explore the natural world.
After a lifetime spent in the Midwest, Nicholas enjoyed living within a short, day-trip distance to many different beaches---a foreign experience to someone from the Heartland.
Nicholas also has a passion for comic books, science fiction, and video games---he is especially interested in the media where those intersect.
This passion for scientific fantasy and video games led him down the path of researching how we could represent real-world physics in simulated environments.
Throughout his PhD, Nicholas learned of his passion for developing creative solutions to technical problems in chemical software. 
Nicholas completed his PhD in the spring of 2025.
Continuing on this path into the research computing industry, he looked forward to working as a computational chemist or solutions architect.
