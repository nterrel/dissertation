\chapter{Concluding Remarks} 
\label{conclusion}

Machine-learned interatomic potential models are a powerful framework for conducting large-scale molecular dynamics simulations with near-quantum accuracy.
Furthering ANI models toward generalization with greater real-world utility required advances in uncertainty quantification, data-efficient training, and computational scalability.
The Early Earth project is a demonstration of the capabilities of ANI models, pushing the boundaries of reactive molecular dynamics with machine-learned potentials to an unprecedented scale.
We reflect in this chapter on the key advancements made throughout this work.

\section{Reflection on Uncertainty in ANI Predictions}

Uncertainty quantification remains a major challenge in the field of machine-learned interatomic potentials.
The ensemble-based approach utilized by ANI models provides a natural means of estimating uncertainty.
Sampling data in active learning via Query by Committee (QBC) of total molecular energy variance was shown to be insufficient for large, complex systems. 
By shifting the focus toward atomic force uncertainty, this work has provided a more physically meaningful approach to assessing model confidence.
The maximum relative deviation in force predictions was shown to be a promising uncertainty metric, offering a scalable way to identify high-error regions of molecules. 
Integrating force-based uncertainty measures directly into the active learning loop would allow ANI models to be trained more efficiently, requiring fewer high-level quantum mechanical calculations while improving model accuracy in underrepresented regions of chemical space.
With the development of LUKE: Use the Forces, a tool for identifying and extracting high-uncertainty atomic environments, we move toward a more targeted approach to data selection. 
Refining training sets based on regions where the model struggles most will push ANI models to greater accuracy with less data, a key step toward scaling machine-learned potentials for broader applications.

\section{Reflection on Early Earth Reactive Simulations}

The Early Earth project proves that ANI potentials are a powerful tool for modeling prebiotic chemistry on a massively distributed scale. 
The Hero Run simulation demonstrates the scaling capabilities of LAMMPS-ANI to tens of millions of atoms and with just 4.4 ns of evolution shows an unprecedented exploration of chemical complexity.
This study opens new avenues for probing the origins of life, but also presents several challenges that future research must address.
A fundamental limitation of simulation on this scale lies in the analysis of trajectory data. 
The data generated by the Hero Run required the development of new GPU-accelerated methods to filter and classify molecules. 
Even with these improvements, identifying all of the meaningful reaction pathways in hundreds of terabytes of data remains a bottleneck. 

The work presented in this dissertation is aimed to demonstrate the scientific value of combining neural network potential models with massively parallel computing to simulate realistic, reactive chemical environments.
From the continued development of uncertainty metrics and atom-level data selection strategies to the deployment of ANI potentials on systems with millions of atoms, the work presented here is only a starting point for the role machine learning models are to play in chemistry moving forward.
