\chapter{Ground state atomic energies}
\label{appendix:GSAEs}
ANI models predict atomic energy outputs that are deviations from the self-energies of each atom. 
For the original 1x and 2x models, self atomic energies (SAEs) were used, while 2xr and other modern ANI models ground state atomic energies (GSAEs). 
Self energies are described in greater detail in Subsection \ref{subsec:total_E_sum_AEs}.


\begin{table}[hb!]
\label{tbl:GSAEs}
\center 
    \caption[$\omega$B97X ground-state atomic energies used in ANI-2xr]{Calculated ground-state atomic energies (GSAEs) for H, C, N, O, S, F, and Cl atoms. All values presented in kcal/mol for consistency with results discussed, though the TorchANI atomic neural networks and EnergyShifter only see values in Hartree.}
    \begin{tabular}{l r}
    \toprule
    Element & Ground state atomic energy \\
    \midrule
    Hydrogen & $-313.3288$ kcal/mol \\
    Carbon & $-23,741.0879$ kcal/mol \\
    Nitrogen & $-34,245.2539$ kcal/mol \\
    Oxygen & $-47,089.8477$ kcal/mol \\
    Sulfur & $-249,799.8594$ kcal/mol \\
    Fluorine & $-62,559.4922$ kcal/mol \\
    Chlorine & $-288,727.5938$ kcal/mol \\
    \bottomrule
    \end{tabular}
\end{table}


\chapter{Medium-sized Early Earth System: Total Interesting Formula Appearances}


\begin{table}[ht!]
\centering
\caption[Molecules with matching formulas found throughout the 228,000 atom run]{Molecules with matching formulas found throughout the medium atom Early Earth simulation; total appearances and the maximum number appearances in a single frame are provided.
}\label{tbl:228k_mol_counts}
\begin{tabularx}{0.775\textwidth}{llrr}  
\toprule
Name & Formula & Total appearances & Max appearances \\
\midrule
Adenine & C5H5N5 & 1  & 1 \\
Guanine & C5H5N5O & 4  & 1 \\ 
Cytosine & C4H5N3O & 11,059 & 3 \\
Thymine & C5H6N2O2 & 26,820 & 3 \\
Uracil & C4H4N2O2 & 30,892 & 3 \\
Glucose & C6H12O6 & 0 & 0 \\
Fructose & C6H12O6 & 0 & 0 \\
Ribose & C5H10O5 & 1 & 1 \\
Deoxyribose & C5H10O4 & 25 & 1 \\
Caprylic acid & C8H16O2 & 87 & 1 \\
Capric acid & C10H20O2 & 0 & 0 \\
Lauric acid & C12H24O2 & 0 & 0 \\
Myristic acid & C14H28O2 & 0 & 0 \\
Palmitic acid & C16H32O2 & 0 & 0 \\
Alanine & C3H7NO2 & 39,376 & 5 \\
Arginine & C6H14N4O2 & 0 & 0 \\
Asparagine & C4H8N2O3 & 495 & 1 \\
Aspartic acid & C4H7NO4 & 187 & 1 \\
Glutamine & C5H10N2O3 & 205 & 1 \\
Glutamic acid & C5H9NO4 & 80 & 1 \\
Glycine & C2H5NO2 & 118,675 & 14 \\
Histidine & C6H9N3O2 & 1,370 & 2 \\
Isoleucine & C6H13NO2 & \multicolumn{2}{r}{(counts mixed with Leucine)} \\
Leucine & C6H13NO2 & 398 & 2 \\
Lysine & C6H14N2O2 & 25 & 1 \\
Phenylalanine & C9H11NO2 & 15,416 & 3 \\
Proline & C5H9NO2 & 42,655 & 4 \\
Serine & C3H7NO3 & 1,718 & 2 \\
Threonine & C4H9NO3 & 464 & 1 \\
Tryptophan & C11H12N2O2 & 1,236 & 2 \\
Tyrosine & C9H11NO3 & 2,888 & 2 \\
Valine & C5H11NO2 & 1,845 & 2 \\
Total: & 31 ref. formulas & 295,927 & matching formulas \\
\bottomrule
\end{tabularx}
\end{table}


\chapter{Customized Topology Loader}
\label{appendix:loader_function}

\begin{lstlisting}
def load_topology(file_name: str):
    """Get the topology out from the file. This function is a slightly reworked version of the topology method from md.load

    Returns
    -------
    topology : mdtraj.Topology
        A topology object
    """
    def get_node(file_name: str, root: str, name: str) -> tp.Any:
        filters = tables.Filters(complib="zlib", shuffle=True, complevel=1)
        f = tables.open_file(file_name, mode="r", filters=filters)
        return f.get_node(root, name=name)
    try:
        raw = get_node(file_name, root="/", name="topology")[0]
        if not isinstance(raw, string_types):
            raw = raw.decode()
        topology_dict = json.loads(raw)
    except tables.NoSuchNodeError:
        return None
    topology = Topology()
    for chain_dict in sorted(topology_dict["chains"], key=operator.itemgetter("index")):
        chain = topology.add_chain()
        for residue_dict in sorted(
            chain_dict["residues"], key=operator.itemgetter("index")
        ):
            try:
                resSeq = residue_dict["resSeq"]
            except KeyError:
                resSeq = None
                warnings.warn(
                    "No resSeq info found in HDF file, defaulting to zero-based idxs"
                )
            try:
                segment_id = residue_dict["segmentID"]
            except KeyError:
                segment_id = ""
            residue = topology.add_residue(
                residue_dict["name"], chain, resSeq=resSeq, segment_id=segment_id
            )
            for atom_dict in sorted(
                residue_dict["atoms"], key=operator.itemgetter("index")
            ):
                try:
                    element = elem.get_by_symbol(atom_dict["element"])
                except KeyError:
                    element = elem.virtual
                topology.add_atom(atom_dict["name"], element, residue)
    atoms = list(topology.atoms)
    for index1, index2 in topology_dict["bonds"]:
        topology.add_bond(atoms[index1], atoms[index2])
    return topology
\end{lstlisting}


\chapter{Molfind GraphMatcher}
\label{appendix:graph_matcher}

The \verb|NetworkX| isomorphism check does not, by default, check the identity of nodes and edges.
This leads to the misidentification of molecules, as shown in Figure \ref{fig:mismatched_alanine}.
To correct this, categorical node and edge matching was added to the \verb|nx.GraphMatcher|, ensuring that every molecule identified via comparison to the PubChem dataset was an exact structural match. \bigskip

\begin{lstlisting}
def update_graph_with_elements(graph):
    """Updates reference graph with element symbols based on atomic numbers."""
    for node, data in graph.nodes(data=True):
        data['element'] = species_dict[data['atomic_number']]
    return graph

def add_element_pairs_to_edges(graph):
    """Adds element pair attributes to the edges of the graph for easier comparison."""
    update_graph_with_elements(graph)
    for (node1, node2) in graph.edges():
        element1 = graph.nodes[node1]['element']
        element2 = graph.nodes[node2]['element']
        graph.edges[node1,node2]['element_pair'] = '-'.join(sorted([element1,element2]))

def edge_match(edge1, edge2):
    """Simple check for element pairs to ensure correct bonding pattern in every molecule that is isomorphic with a reference grah."""
    return edge1['element_pair'] == edge2['element_pair']            
\end{lstlisting}

This correction is implemented into the loop where graph comparisons are made, just before the graph of a molecular fragment that was found in the simulation trajectory is compared to a reference graph stored in the PubChem dataset. \bigskip

\begin{lstlisting}
add_element_pairs_to_edges(fragment_graph)
add_element_pairs_to_edges(graph)
node_match = nx.isomorphism.categorical_node_match('element', '')
edge_match = nx.isomorphism.categorical_edge_match('element_pair', '')
gm = nx.isomorphism.GraphMatcher(graph, fragment_graph,
                    node_match=node_match, edge_match=edge_match)
\end{lstlisting}


\chapter{Formula Matches Versus Structural Matches}
\label{appendix:frag_vs_mol}

\begin{table}[h!]
\centering
\caption[Comparison of fragment formulas matching a reference formula to structural matches]{Comparison of total formulas matching a reference formula to the actual structural matches identified after the graph comparison across the entire Hero Run simulation.
}\label{tbl:formula_vs_mol}
\begin{tabularx}{0.6125\textwidth}{lrlr}  
\toprule
Formula & Matches & Name & Matches \\
\midrule
\ce{C5H5N5} & 2,086 & Adenine & 0 \\
\ce{C3H7NO2} & 5,628,694 & Alanine & 445,342 \\
\ce{C5H10N2O3} & 25,038 & Alanylglycine & 0 \\
\ce{C6H14N4O2} & 293 & Arginine & 0 \\
\ce{C4H8N2O3} & 71,916 & Asparagine & 971 \\
\ce{C4H7NO4} & 35,248 & Aspartic Acid & 439 \\
\ce{C10H20O2} & 298 & Capric acid & 0 \\
\ce{C8H16O2} & 10,562 & Caprylic acid & 137 \\
\ce{C4H5N3O} & 1,570,671 & Cytosine & 227 \\
\ce{C5H10O4} & 2,976 & Deoxyribose & 0 \\
\ce{C5H9NO4} & 10,897 & Glutamic Acid & 37 \\
\ce{C5H10N2O3} & 25,038 & Glutamine & 42 \\
\ce{C2H5NO2} & 19,724,180 & Glycine & 5,297,484 \\
\ce{C4H8N2O3} & 71,916 & Glycylglycine & 14 \\
\ce{C5H5N5O} & 1,269 & Guanine & 0 \\
\ce{C6H9N3O2} & 215,761 & Histidine & 0 \\
\ce{C6H13NO} & 59,855 & Isoleucine & 30 \\
\ce{C12H24O2} & 7 & Lauric acid & 0 \\
\ce{C6H13NO} & 59,855 & Leucine & 46 \\
\ce{C6H14N2O2} & 5,877 & Lysine & 10 \\
\ce{C9H11NO2} & 2,377,381 & Phenylalanine & 0 \\
\ce{C5H9NO} & 6,884,429 & Proline & 0 \\
\ce{C5H10O5} & 109 & Ribose & 0 \\
\ce{C3H7NO3} & 227,175 & Serine & 2,183 \\
\ce{C4H9NO3} & 67,953 & Threonine & 202 \\
\ce{C5H6N2O2} & 3,271,455 & Thymine & 9 \\
\ce{C11H12N2O2} & 253,587 & Tryptophan & 0 \\
\ce{C9H11NO3} & 461,133 & Tyrosine & 0 \\
\ce{C4H4N2O2} & 3,907,529 & Uracil & 414 \\
\ce{C5H11NO} & 284,861 & Valine & 591 \\
Total & 45,198,194 & Total & 5,748,178 \\
\bottomrule
\end{tabularx}
\end{table}