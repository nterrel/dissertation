\chapter{Future work} 
\label{future_work}

While the work presented in this dissertation has demonstrated significant advancements in machine-learned potentials, uncertainty quantification, and large-scale molecular simulations, several promising directions remain for future research.
In particular, targeted uncertainty-driven data sampling stands out as an important next step in refining ANI potentials for broader applications.
LUKE: Use the Forces provides a targeted approach for identifying and extracting high-uncertainty atomic environments. 
The implementation of a trajectory analysis protocol within LUKE would provide a near-limitless source of new molecular configurations from the Hero Run trajectory. 

Future directions for the Hero Run project should focus on a more robust characterization of the chemical diversity sampled throughout the simulation.
Expanding the PubChem dataset to include more molecules of biological or prebiotic relevance is helpful in the current data analysis pipeline, but the ideal solution would be to characterize all unique molecular species synthesized throughout the simulation.
There is currently an ongoing effort in GPU-accelerated graph analysis libraries to implement the unique graph operations supported by their CPU counterparts. 
With graph comparison algorithms capable of running on GPU, it would be possible to store every unique molecular fragment that shows up throughout this simulation.
Characterizing one example of each of the unique molecular species identified throughout the trajectory is far more feasible than on-the-fly characterization of every fragment in each frame of a simulation with millions of atoms. 

The Hero Run strategy of conducting large-scale reactive simulations should be utilized in building datasets with increased chemical diversity (\textit{i.e.}, more elements).
Reactive MD with potential models that support a larger set of elements, along with on-the-fly uncertainty-based sampling, could aid in expanding the ANI datasets to new atom types, particularly phosphorus.
These strategies would push ANI models toward a more autonomous and chemically informed framework for model improvement, where large-scale simulations are used to refine machine-learned potentials in a continuous loop. 
