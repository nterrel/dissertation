\chapter{Future work} 
\label{future_work}

While this dissertation has demonstrated significant advancements in machine-learned potentials, uncertainty quantification, and large-scale molecular simulations, several promising directions remain for future research. In particular, two key areas—targeted uncertainty-driven data augmentation and leveraging large-scale reactive simulations for model improvement—stand out as crucial next steps toward refining ANI potentials for broader applications.

The Atom Isolator protocol, developed in this work, provides a targeted approach for identifying and extracting high-uncertainty atomic environments. Instead of retraining ANI models on entire molecules, which often contain well-learned regions, Atom Isolator selectively isolates the atomic substructures that contribute most to model uncertainty. This method could be further refined to systematically sample new structures by (1) Expanding the fragment selection criteria to include chemically diverse environments beyond simple cutoff-based truncation. (2) Developing a protocol for capping isolated fragments with appropriate terminal groups, ensuring that extracted structures retain chemically meaningful interactions. (3) Incorporating Atom Isolator into an active learning framework, allowing for iterative dataset expansion focused on regions where ANI models currently struggle and expanding the ANI datasets into a more diverse set of elements.
By applying this approach to diverse chemical spaces, ANI models could become more data-efficient, learning from smaller but more informative datasets rather than requiring exponentially more full-molecule training examples.

The Early Earth simulation has provided an extensive dataset of chemically diverse molecules formed under prebiotic conditions. While our initial analysis focused on identifying broad classes of organic compounds, future work should expand the search criteria to include more biologically relevant molecules. This includes not only amino acids and simple sugars but also nucleobase precursors, lipid-like structures, and small peptides—key molecular building blocks associated with the origins of life.

A critical next step is to refine and expand the molecular graph search, incorporating additional molecular patterns that reflect the core functional groups of biologically significant molecules. This could be achieved by developing an extended molecular fingerprinting approach that prioritizes structures containing peptide bonds and heterocyclic frameworks, expanding the search to detect reaction networks and identifying sequences of chemical transformations that lead to progressively more complex molecular assemblies. In addition to expanding the molecular search space, future work should also investigate the spatial and temporal distribution of key molecular species within the simulation. By mapping where and when certain biomolecules emerge, we can gain deeper insight into whether specific environmental conditions favor the formation of biologically relevant structures.

These improvements will allow us to extract chemical insights from the Early Earth dataset, bridging the gap between ab initio molecular simulations and experimental prebiotic chemistry. Ultimately, this will help us better understand how fundamental biological building blocks could have emerged from primitive Earth-like conditions, refining our models of early molecular evolution.