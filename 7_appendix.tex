\chapter{Neural network structures in ANAKIN-ME models}

Give the shape and number of parameters of ANI networks (at least 1x, 1ccx, 2x, DR, and 1xnr since these are all being used or referenced in this dissertation. 

Format this nicely:
ANI-2x: 8 ensembled models \\
AEVComputer: outputs fixed-length vector encoding atomic environment \\
StandardAngular, StandardRadial; CutoffCosine (for radial+angular): determine terms and AEV radius \\
FullPairwise: neighbor list computation\\
EnergyShifter: converts energies from near-zero values output by networks to the "physical range" of values for different atom types \\
ChemicalSymbolsToInts, SpeciesConverter: helper functions to produce species tensor \\
H Network shape: 1008 x 256 x 192 x 160 x 1 \\
C Network shape: 1008 x 224 x 192 x 160 x 1 \\
N Network shape: 1008 x 192 x 160 x 128 x 1 \\
O Network shape: 1008 x 192 x 160 x 128 x 1 \\
S Network shape: 1008 x 160 x 128 x 96 x 1 \\
F Network shape: 1008 x 160 x 128 x 96 x 1 \\
Cl Network shape: 1008 x 160 x 128 x 96 x 1 \\

ANIdr: 7 ensembled models \\
Pair interactions: AEVComputer (FullPairwise: [StandardAngular, CutoffSmooth]; [StandardRadial, CutoffSmooth]) \\
EnergyShifter, ChemicalSymbolsToInts, SpeciesConverter same as 2x \\
Linear layers connected with GELU activation functions: \\
H Network shape: 1008 x 256 x 192 x 160 x 1 \\
C Network shape: 1008 x 224 x 192 x 160 x 1 \\
N Network shape: 1008 x 192 x 160 x 128 x 1 \\
O Network shape: 1008 x 192 x 160 x 128 x 1 \\
S Network shape: 1008 x 160 x 128 x 96 x 1 \\
F Network shape: 1008 x 160 x 128 x 96 x 1 \\
Cl Network shape: 1008 x 160 x 128 x 96 x 1 \\
TwoBodyDispersionD3 (CutoffSmooth, BJDamp) \\
RepulsionXTB (CutoffSmooth) \\
Additional network architecture for repulsion, linear layers connected with GELU activation functions: \\
H Network shape: 1008 x 256 x 192 x 160 x 1 \\
C Network shape: 1008 x 224 x 192 x 160 x 1 \\
N Network shape: 1008 x 192 x 160 x 128 x 1 \\
O Network shape: 1008 x 192 x 160 x 128 x 1 \\
S Network shape: 1008 x 160 x 128 x 96 x 1 \\
F Network shape: 1008 x 160 x 128 x 96 x 1 \\
Cl Network shape: 1008 x 160 x 128 x 96 x 1 \\


\chapter{Description of AEV parameters}

Appendix E in Richard's dissertation could be a useful reference, but double check with GitHub for updated values.

\chapter{Supplemental information}
{\bf Paragraph headings.} There is no official fourth level heading. Do not use the Paragraph heading feature in LaTeX, simply apply the bold characteristic to the first few words of a paragraph followed by a colon or period.

Please note that if you are adding a table or figure that takes up a whole page, you must have some text underneath your appendix title before using the next page.

\begin{table}[h]
\caption[Appendix table]{Additionally, the rules regarding italics and bolding does not apply to appendices. You can do whatever you'd like here mostly.}
%\begin{center}
\begin{tabularx}{\textwidth}{XXXX}\hline
\textbf{Some}    & \textbf{Data}  & \textbf{Goes}  & \textbf{Here}\\\hline
Some    & Data  & Goes  & Here\\
Some    & Data  & Goes  & Here\\
Some    & Data  & Goes  & Here\\\hline
\end{tabularx}
%\end{center}

\end{table}